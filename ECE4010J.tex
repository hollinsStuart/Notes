\documentclass[11pt,a4paper]{article}

%just apply these packets
\usepackage{amsmath}
\usepackage{amssymb}
\usepackage{amsfonts}
\usepackage{amsthm}     %for Proof environment
\usepackage{enumerate}
\usepackage{geometry}
\geometry{top=1in,bottom=1in,left=1in,right=1in}
\usepackage{tikz}
\usepackage{pgfplots}
\pgfplotsset{height=10cm,compat=1.18}
\usepackage{empheq}
\usepackage{xcolor}

\newcommand{\uu}{\mathrm{u}}
\newcommand{\dd}{\mathrm{d}}
\newcommand{\dt}{\mathrm{d}t}
\newcommand{\ddt}{\frac{\dd}{\dd t}}
\newcommand{\dA}{\dd A}
\newcommand{\ii}{\mathbf{i}}
\newcommand{\jj}{\mathbf{j}}
\newcommand{\kk}{\mathbf{k}}
\newcommand{\dx}{\dd x}
\newcommand{\ddx}{\ddot{x}}
\newcommand{\dy}{\dd y}
\newcommand{\ddy}{\ddot{y}}
\newcommand{\dz}{\dd z}
\newcommand{\ddz}{\ddot{z}}
\newcommand{\ppx}{\dfrac{\partial }{\partial x}}
\newcommand{\ppy}{\dfrac{\partial }{\partial y}}
\newcommand{\ppz}{\dfrac{\partial }{\partial z}}
%\newcommand{\det}{\mathrm{det}}
\newcommand{\sgn}{\mathrm{sgn}(\pi)}
\newcommand{\TT}{\mathbf{T}}
\newcommand{\NN}{\mathbf{N}}
\newcommand{\BB}{\mathbf{B}}
\newcommand{\dydx}{\dfrac{\dd y}{\dd x}}
\newcommand{\tc}{\tilde{c}}

\newtheorem*{theorem}{THEOREM}
\newtheorem*{lemma}{LEMMA}
\newtheorem*{proposition}{PROPOSITION}
\newtheorem*{corollary}{Corollary}
\newtheorem*{example}{\bf EXAMPLE}
\newtheorem*{definition}{\bf DEFINITION}

\begin{document}
\author{Hollins Yu}
\title{\bf  ECE4010J \\ \LARGE Probabilistic Methods in Engineering}
\maketitle
\newpage
\tableofcontents
\newpage
\noindent
{\fontsize{40pt}{20}\bf\color[RGB]{255,0,0}DO\\NOT\\GAMBLE}
\newpage

\part{Elements of Probability Theory}
\section{Elementary Probility}

\subsection{Basic Principles of Counting - Permuutation and Combination}

\subsection{Binomial Coefficients}
We define binomial coefficients by
\begin{gather}
    \binom{\alpha}{0} := 1 \\
    \binom{\alpha}{n}:=\frac{\alpha\cdot(\alpha-1)\cdots(\alpha-n+1)}{n!}
\end{gather}

\section{Conditional Probability}


\subsection{Total Probability}
Let $A_i$ be a set of mutually exclusive events that $\bigcup A_i=S$.
Let $B\subset S$ be any event, then the probability of $B$ is
\begin{gather}
    P[B]=\sum_{k=1}^{n}P[B|A_k]\cdot P[A_k].
\end{gather}
\subsection{Bayer's Theorem}
Let $A_i$ be a set of mutually exclusive events that $\bigcup A_i=S$.
Let $B\subset S$ be any event, then the probability of $B$ is
\begin{gather}
    P[A_k|B]=\frac{P[B\cap A_k]}{P[B]}=\frac{P[B| A_k]\cdot P[A_k]}{\sum_{j=1}^{n}P[B| A_j]\cdot P[A_j]}.
\end{gather}

\begin{example}
    A Rare Disease
\end{example}
\begin{example}
    The Monty Hall Paradox
\end{example}


\section{Discrete Random Varaibles}
\subsection{Bernoulli Trail}
\begin{definition}
    Consider an event with success probability $p\in(0,1)$. Then the trail is called
    a {\bf Bernoulli trail}.
    \begin{align}
        f_X:\{0,1\}\rightarrow \mathbb{R}, \quad f_X(x)=\begin{cases*}
                                                            1-p & {\rm for} x = 0  \\
                                                            p   & {\rm for} x = 1.
                                                        \end{cases*}
    \end{align}
    Then $X$ is a Bernoulli random variable, denoted as\[X\sim \mathrm{Bernoulli}(p).\]
\end{definition}

\subsection{Binomial Random Variable}
To count a sepuence of identical Bernoulli trails, we find\[P[x \mathrm{\ successes\ in\ } n \mathrm{\ trails}]=
    \binom{n}{x}p^x(1-p)^{n-x}.\]
\begin{definition}
    Let $0<p<1$ and define the density function
    \begin{align}
        f_X:\varOmega \rightarrow \mathbb{R},
        \quad f_X(x)=\binom{n}{x}p^x(1-p)^{n-x}.
    \end{align}
    Then $X$ is a Binomial random variable, denoted as\[X\sim \mathrm{B}(n,p).\]
\end{definition}

\subsection{The Geometric Distribution}
\[X: S\rightarrow\Omega = \mathbb{N} \backslash \{0\}\]

\begin{example}
    Find the probability that a roulette wheel returned black more than 20 times in a
    row.
\end{example}


\noindent{\bf Solution:}

123


\section{Expectation, Variance and Moments}
\subsection{Expection}
The expectation of $X$ is:
\[\mathrm{E}[X]:=\sum_{x\in\Omega}^{}x\cdot f_X(x)\]
\subsection{Functions of Random Variables}


\subsection{Varience and Standard Deviation}
*some DIY proof needed here*


\section{The Pascal, Negative Binomial and Poisson Distributions}
\subsection{The Pascal Distribution}

\subsection{The Negative Binomial Distribution}

\subsection{The Poisson Distribution}
The Poisson distribution describes the occurrence of events that occur
at a {\it\bf constant rate} in a {\it\bf continuous} environment.
It's density function and moment generating function are given by
\begin{align}
    f_X(x) = \frac{k^x e^{-k}}{x!} \\
    m_X  (t) = e^{k(e^t-1)}
\end{align}



\subsection{Approximating the Binomial Distribution}
With a large enough number of trails, we can
If $n\rightarrow\infty$ while $n\cdot p=:\lambda$ remains constant,
\begin{gather}
    \binom{n}{m}
    p^m(1-p)^{n-m}     \xrightarrow[n\cdot p =k]{n\rightarrow \infty} \frac{k^m}{m!}e^{-k}
\end{gather}

\section{Continuous Random Variables}
\noindent {\large \textsc{Definition.}}
A continuous random variable $(X,f_X)$
\begin{enumerate}
    \item $f_X\le0$
    \item $\int_{-\infty}^{\infty}f_X\ \dx = 0$
\end{enumerate}
*Notice: Probability of any specific value is 0.
\subsection{Cumulative Distribution}
\begin{gather}
    F_X(x):=P[X\le x]=\int_{-\infty}^{x}f_X(y)\ \dy\\
    \mathrm{E}[X]:=\int_\mathbb{R}x\cdot F_X(x)\ \dx\\
    \mathrm{E}[\varphi \circ x] = \int^\infty_{-\infty}\varphi(x)f_X(x)\ \dx
\end{gather}

\subsection{The Exponential Distribution}
\begin{gather}
    f_\beta(x) = \begin{cases}
        \beta e^{-\beta x} & x>0   \\
        0                  & x\le0
    \end{cases}
\end{gather}
Expectation:
\begin{align*}
    \mathrm{E}[X] & = \int_{-\infty}^{\infty}x\cdot f_\beta(x)\ \dx   \\
                  & = \int_{0}^{\infty}x\cdot \beta e^{-\beta x}\ \dx \\
                  & =
\end{align*}


\subsection{Location of Continuous Distributions}
i
ii
iii

\section{The Normal Distribution}
Let $\mu\in\mathbb{R}$, $\sigma>0$. A continuous random variable $(X,f_X)$ with density function
\[f_X(x) = \frac{1}{\sqrt{2\pi}\sigma}e^{-((x-\mu)/\sigma^2)/2}\]
follows the normal distribution.
We denote as $X\sim N(\mu,\sigma)$ if $X$ follows normal distribution with mean
$\mu$ and variance $\sigma^2$.
\newline

\begin{proposition}
    $$\int_{\mathbb{R}}f_X(x)\ \dx=1$$
\end{proposition}
\begin{proof}
    Let $y=\dfrac{x-\mu}{\sigma}$, then we get
    \[\frac{1}{\sqrt{2\pi}}\int_{-\infty}^{\infty}e^{-y^2/2}\ \dy\]
    Then we calculate the squared value.
    \begin{align*}
        \left(\frac{1}{\sqrt{2\pi}}\int_{-\infty}^{\infty}e^{-y^2/2}\ \dy\right)^2 & =
        \frac{1}{2\pi}\iint_{\mathbb{R}^2}e^{-(x^2+y^2)/2}\,\dx\,\dy                   \\
    \end{align*}
    Use polar coordinates to find the final value.
    \begin{align*}
         & =\frac{1}{2\pi}\int_{0}^{\infty}\int_{0}^{2\pi}e^{-r^2/2}r\ \dd\theta\ \dd r \\
         & = 1
    \end{align*}
\end{proof}
{\bf Find the M.G.F of the normal distribution.}


\subsection{Standard Normal Distribution}
A normal distribution with $\mu=0$ and $\sigma=1$ is called {\it standard normal}
random variable, denoted by $Z$.

\subsection{Transformation of Random Variables}
Let $X$ be a continuous random variable with density $f_X$, and $Y=\varphi\circ X$,
where $\varphi:\mathbb{R}\rightarrow\mathbb{R}$ is strictly monotonic and differentiable.
The density for $Y$ is then given by
\[f_Y(y)=\]

\section{Multivariate Random Variables}
\section{The Weak Law of Large Numbers}
\section{The Hypergeometric Distribution}
\section{Transformation of Random Variables}
\section{Reliability}

\newpage
\part{Introduction to Stastics}
\section{Samples and Data}
\section{Parameter Estimation}
\section{Interval Estimation}
\section{The Fisher Test}
\section{Neyman-Pearson Decision Theory}
\section{Null Hypothesis Significance Testing}
\section{Single Sample Tests for the Mean and Variance}
\section{Non-Parametric Single Sample Tests for the Median}
\section{Inferences on Proportions}
\section{Comparison of Two Variances}
\section{Comparison of Two Means}
\section{Non-Parametric Comparisons; Paired Tests and Correlation}
\section{Categorical Data}

\newpage
\part{Introduction to Linear Regression}
\section{Simple Linear Regression I: Basic Model and Inferences}
\section{Simple Linear Regression II: Predictions and Model Analysis}
\section{Multiple Linear Regression I: Basic Model}
\section{Multiple Linear Regression II: Inferences on the Model}
\section{Multiple Linear Regression III: Finding the Right Model}
\section{ANOVA I: Basic Model}
\section{ANOVA II: Homoscedasticity and Post-Tests}


\end{document}
