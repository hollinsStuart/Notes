\documentclass[11pt,a4paper]{article}

%just apply these packets
\usepackage{amsmath}
\usepackage{amssymb}
\usepackage{amsfonts}
\usepackage{amsthm}     %for Proof environment
\usepackage{enumerate}
\usepackage{geometry}
\geometry{top=1in,bottom=1in,left=1in,right=1in}
\usepackage{tikz}
\usepackage{pgfplots}
\pgfplotsset{height=10cm,compat=1.18}
\usepackage{empheq}
\usepackage{xcolor}

\newcommand{\uu}{\mathrm{u}}
\newcommand{\dd}{\mathrm{d}}
\newcommand{\dt}{\mathrm{d}t}
\newcommand{\ddt}{\frac{\dd}{\dd t}}
\newcommand{\dA}{\dd A}
\newcommand{\ii}{\mathbf{i}}
\newcommand{\jj}{\mathbf{j}}
\newcommand{\kk}{\mathbf{k}}
\newcommand{\dx}{\dd x}
\newcommand{\ddx}{\ddot{x}}
\newcommand{\dy}{\dd y}
\newcommand{\ddy}{\ddot{y}}
\newcommand{\dz}{\dd z}
\newcommand{\ddz}{\ddot{z}}
\newcommand{\ppx}{\dfrac{\partial }{\partial x}}
\newcommand{\ppy}{\dfrac{\partial }{\partial y}}
\newcommand{\ppz}{\dfrac{\partial }{\partial z}}
%\newcommand{\det}{\mathrm{det}}
\newcommand{\sgn}{\mathrm{sgn}(\pi)}
\newcommand{\TT}{\mathbf{T}}
\newcommand{\NN}{\mathbf{N}}
\newcommand{\BB}{\mathbf{B}}
\newcommand{\dydx}{\dfrac{\dd y}{\dd x}}
\newcommand{\tc}{\tilde{c}}
\newcommand{\XX}{\mathbf{X}}
\newcommand{\YY}{\mathbf{Y}}

\newtheorem*{theorem}{\bf THEOREM}
\newtheorem*{lemma}{\bf LEMMA}
\newtheorem*{proposition}{\bf PROPOSITION}
\newtheorem*{corollary}{\bf Corollary}
\newtheorem*{example}{\bf EXAMPLE}
\newtheorem*{definition}{\bf DEFINITION}

\begin{document}
\author{Hollins Yu}
\title{\bf  ECE4010J \\ \LARGE Probabilistic Methods in Engineering}
\maketitle

\newpage
\tableofcontents

\newpage
\noindent
{\fontsize{40pt}{20}\bf\color[RGB]{255,0,0}DO\\NOT\\GAMBLE}
\newpage

\part{Elements of Probability Theory}
\section{Elementary Probility}

\subsection{Basic Principles of Counting - Permuutation and Combination}

\subsection{Binomial Coefficients}
We define binomial coefficients by
\begin{gather}
    \binom{\alpha}{0} := 1 \\
    \binom{\alpha}{n}:=\frac{\alpha\cdot(\alpha-1)\cdots(\alpha-n+1)}{n!}
\end{gather}

\section{Conditional Probability}
\subsection{Total Probability}
Let $A_i$ be a set of mutually exclusive events that $\bigcup A_i=S$.
Let $B\subset S$ be any event, then the probability of $B$ is
\begin{gather}
    P[B]=\sum_{k=1}^{n}P[B|A_k]\cdot P[A_k].
\end{gather}
\subsection{Bayer's Theorem}
Let $A_i$ be a set of mutually exclusive events that $\bigcup A_i=S$.
Let $B\subset S$ be any event, then the probability of $B$ is
\begin{gather}
    P[A_k|B]=\frac{P[B\cap A_k]}{P[B]}=\frac{P[B| A_k]\cdot P[A_k]}{\sum_{j=1}^{n}P[B| A_j]\cdot P[A_j]}.
\end{gather}

\begin{example}
    A Rare Disease
\end{example}
\begin{example}
    The Monty Hall Paradox
\end{example}

\section{Discrete Random Varaibles}
\begin{definition}
    The probability density function or probability
    distribution of X is defined as \[f_X(x)=P[X=x]\]
\end{definition}
\subsection{Bernoulli Trail}
\begin{definition}
    Consider an event with success probability $p\in(0,1)$. Then the trail is called
    a {\bf Bernoulli trail}.
    \begin{align}
        f_X:\{0,1\}\rightarrow \mathbb{R}, \quad f_X(x)=\begin{cases*}
                                                            1-p & {\rm for} x = 0  \\
                                                            p   & {\rm for} x = 1.
                                                        \end{cases*}
    \end{align}
    Then $X$ is a Bernoulli random variable, denoted as\[X\sim \mathrm{Bernoulli}(p).\]
\end{definition}

\subsection{Binomial Random Variable}
To count a sepuence of identical Bernoulli trails, we find\[P[x \mathrm{\ successes\ in\ } n \mathrm{\ trails}]=
    \binom{n}{x}p^x(1-p)^{n-x}.\]
\begin{definition}
    Let $0<p<1$ and define the density function
    \begin{align}
        f_X:\varOmega \rightarrow \mathbb{R},
        \quad f_X(x)=\binom{n}{x}p^x(1-p)^{n-x}.
    \end{align}
    Then $X$ is a Binomial random variable, denoted as\[X\sim \mathrm{B}(n,p).\]
\end{definition}

\subsection{Cumulative Distribution Function}
In practice, the cumulative distribution function is defined as
\[F_X:\mathbb{R}\rightarrow\mathbb{R}, \quad
    F_X(x):=P[X\le x]\]
In mathmatica, we calculate C.D.F using the following command
\begin{verbatim}
    CDF[BinomialDistribution[n,p],x]
\end{verbatim}

\subsection{The Geometric Distribution}
\label{Geom}
When counting the first success in a sequence of Bernoulli trails,
we introduce the Geometric distribution.
\begin{definition}
    Let $S$ be the sample space, $0<p<1$.
    \begin{gather*}
        X: S\rightarrow\varOmega = \mathbb{N} \backslash \{0\} \\
        f_X:\mathbb{N}\backslash \{0\}\rightarrow\mathbb{R},
        \quad f_X(x)=(1-p)^{x-1}p
    \end{gather*}
    We say that $X$ is a geometric random variable, denoted as
    \[X\sim \mathrm{Geom}(p).\]
\end{definition}
In mathmatica,
\begin{verbatim}
    PDF[GeometricDistribution[p], x]
    Probability[1 < x ≤ 4, x È GeometricDistribution[p]]
\end{verbatim}
%%%%%%%%%%%%%%%%%%%%%%%%%%%%%%%%%%%%%%%%%%%%%%%%%%%%%%%%%%%%
{\bf\color[RGB]{255,0,0}NOTICE:}

The geometric distribution here
gives the number of {\bf\color[RGB]{255,0,0}failures} before the first success.
\begin{example}
    Find the probability that a roulette wheel returned black more than 20 times in a
    row.
\end{example}


\noindent{\bf Solution:}

123

\section{Expectation, Variance and Moments}
\subsection{Expection}
\begin{definition}
    The expectation of discrete random variable $X$ is:
    \[\mathrm{E}[X]:=\sum_{x\in\varOmega}^{}x\cdot f_X(x).\]
    We write $\mu$ or $\mu_X$ for the expectation.
\end{definition}

\subsection{Functions of Random Variables}
Let $(Y,f_Y)$ be a discrete random variable.
Consider a new random variable $(Y,f_Y)$ that satisfies
\begin{gather*}
    Y=\varphi(X),\quad \varphi:\varOmega\rightarrow\mathbb{R}\\
    f_Y(y)=P[Y=y]=P[\varphi(X)=y]
\end{gather*}
Since $X$ is discrete,
\[P[\varphi(X)=y]=\sum_{\substack{x\in\varOmega\\\varphi(x)=y}}f_X(x).\]


\begin{lemma}
    Let $(X,f_X)$ be a discrete random variable, $\varphi:\varOmega\rightarrow\mathbb{R}$.
    The expection of $\varphi\circ X$ is
    \[\mathrm{E}[\varphi\circ X]=\sum_{x\in\varOmega}^{}\varphi(x)\cdot f_X(x)\]
    if the right hand side converges.
\end{lemma}


\subsection{Varience and Standard Deviation}
\begin{definition}
    The mean square deviation is
    \[\mathrm{Var}[X]:=\mathrm{E}[(X-\mathrm{E}[X])^2]\]
    The variance is denoted by $\sigma_X^2$ or $\sigma^2$.

    The standard deviation is defined as\[\sigma_X=\sqrt{\mathrm{Var}[X]}\]
\end{definition}

\subsection{Standardized Random Variables}
\begin{definition}
    If $X$ is a given random variable, the standardized variable is
    \[Y=\frac{X-\mu}{\sigma}.\]
    We find that $\mathrm{E}[Y]=0$, $\mathrm{Var}[Y]=1$
\end{definition}
\begin{proof}
    *some DIY proof needed here*
\end{proof}

\subsection{Ordinary and Central Moments}
\begin{definition}
    Given a random variable $X$, $\mathrm{E}[X^n]$ are known as the $n^{th}$ ordinary moments of $X$.
    $\mathrm{E}\left[\left(\dfrac{X-\mu}{\sigma}\right)^n\right]$ are called the $n^{th}$ central moments of $X$.
\end{definition}

\subsection{The Moment-Generating Function}
\begin{definition}
    Let $(X,f_X)$ be a random variable with moments $\mathrm{E}[X^n]$.

    If power series
    \[m_X(t):=\sum_{k=0}^{\infty}\frac{\mathrm{E}[X^k]}{k!}t^k\]
    \indent has radius of converge $\varepsilon>0$, then $m_X(t)$ is the M.G.F of $X$.
\end{definition}
\begin{theorem}
    \[m_X(t)=\mathrm{E}[e^{tX}]\quad \mathrm{for}\,|t|<\varepsilon\]
\end{theorem}

\subsection{M.G.F for the Geometric Distribution}

%chapter 5
\section{The Pascal, Negative Binomial and Poisson Distributions}
\subsection{The Pascal Distribution}
The probability of getting $r$ successes in $x$ trails.
\begin{definition}
    The density function of a Pascal Distribution is
    \begin{gather*}
        f_X(x)=\binom{x-1}{r-1}p^r(1-p)^{x-r}
    \end{gather*}
\end{definition}
\subsubsection{The M.G.F for the Pascal Distribution}
\begin{gather*}
    m_X(t)=\frac{(pe^t)^r}{(1-qe^t)^r}, \quad q=1-p\\
    \mathrm{E}[X]=r/p\\
    \mathrm{Var}[X]=rq/p^2
\end{gather*}
\begin{verbatim}
    MomentGeneratingFunction[PascalDistribution[r, p], t]
\end{verbatim}

\subsection{The Negative Binomial Distribution}
Count the number of failures before $r$ successes.
\begin{definition}
    \begin{gather*}
        f_X(x)=\binom{x+r-1}{r-1}p^r(1-p)^x
    \end{gather*}
\end{definition}

\subsection{The Poisson Distribution}
The Poisson distribution describes the occurrence of events that occur
at a {\bf constant rate} in a {\bf continuous} environment.
It's density function and moment generating function are given by
\begin{gather*}
    f_X(x) = \frac{k^x e^{-k}}{x!} \\
    m_X  (t) = e^{k(e^t-1)}\\
    \mathrm{E}[X]=k\\
    \mathrm{Var}[X]=k\\
    F(x)=P[X\le x]=\sum_{y=1}^{\left\lfloor x \right\rfloor}\frac{e^{-k}k^y}{y!}
\end{gather*}

\subsection{Approximating the Binomial Distribution}
With a large enough number of trails, we can
If $n\rightarrow\infty$ while $n\cdot p=:\lambda$ remains constant,
\begin{gather*}
    \binom{n}{m}
    p^m(1-p)^{n-m}     \xrightarrow[n\cdot p =k]{n\rightarrow \infty} \frac{k^m}{m!}e^{-k}
\end{gather*}

% chapter 6
\section{Continuous Random Variables}
\begin{definition}
    A continuous random variable $(X,f_X)$
    \begin{enumerate}
        \item $f_X\le0$
        \item $\int_{-\infty}^{\infty}f_X\ \dx = 0$
    \end{enumerate}
\end{definition}

*Notice: Probability of any specific value is 0.
\subsection{Cumulative Distribution}
\begin{gather}
    F_X(x):=P[X\le x]=\int_{-\infty}^{x}f_X(y)\ \dy\\
    \mathrm{E}[X]:=\int_\mathbb{R}x\cdot F_X(x)\ \dx\\
    \mathrm{E}[\varphi \circ x] = \int^\infty_{-\infty}\varphi(x)f_X(x)\ \dx
\end{gather}

\subsection{The Exponential Distribution}
\begin{gather}
    f_\beta(x) = \begin{cases}
        \beta e^{-\beta x} & x>0   \\
        0                  & x\le0
    \end{cases}
\end{gather}
Expectation:
\begin{align*}
    \mathrm{E}[X] & = \int_{-\infty}^{\infty}x\cdot f_\beta(x)\ \dx   \\
                  & = \int_{0}^{\infty}x\cdot \beta e^{-\beta x}\ \dx \\
                  & = \frac{1}{\beta}
\end{align*}

\subsection{Location of Continuous Distributions}
\begin{enumerate}
    \item Median $M_X$ is defined by $P[M\le M_X]=0.5$.
    \item Mean $\mathrm{E}[X]$.
    \item Mode $x_0$
\end{enumerate}

\subsection{The Gamma Distribution}
\begin{definition}
    A continuous random variable
    $(X, f_{\alpha,\beta})$ with density
    \begin{gather*}
        f_{\alpha,\beta}(x)=\begin{cases}
            \frac{\beta^\alpha}{\Gamma(\alpha)}x^{\alpha-1}e^{-\beta x}, & x>0    \\
            0,                                                           & x\le 0
        \end{cases}
    \end{gather*}
    in which
    \begin{gather*}
        \Gamma(\alpha)=\int_{0}^{\infty}z^{\alpha-1}e^{-z}\,\dz, \quad \alpha>0
    \end{gather*}

\end{definition}
The Gamma function satisfies $\Gamma(1)=1,\Gamma(\alpha)=(\alpha-1)\Gamma(\alpha-1)$

\begin{gather*}
    m_X(t)=(1-t/\beta)^{-\alpha}\\
    \mathrm{E}[X]=\alpha/\beta\\
    \mathrm{Var}[X]=\alpha/\beta^2
\end{gather*}

\section{The Normal Distribution}
\begin{definition}
    Let $\mu\in\mathbb{R}$, $\sigma>0$. A continuous random variable $(X,f_X)$ with density function
    \[f_X(x) = \frac{1}{\sqrt{2\pi}\sigma}e^{-((x-\mu)/\sigma^2)/2}\]
    follows the normal distribution.
    We denote as $X\sim N(\mu,\sigma)$ if $X$ follows normal distribution with mean
    $\mu$ and variance $\sigma^2$.
\end{definition}

\begin{proposition}
    $$\int_{\mathbb{R}}f_X(x)\ \dx=1$$
\end{proposition}
\begin{proof}
    Let $y=\dfrac{x-\mu}{\sigma}$, then we get
    \[\frac{1}{\sqrt{2\pi}}\int_{-\infty}^{\infty}e^{-y^2/2}\ \dy\]
    Then we calculate the squared value.
    \begin{align*}
        \left(\frac{1}{\sqrt{2\pi}}\int_{-\infty}^{\infty}e^{-y^2/2}\ \dy\right)^2 & =
        \frac{1}{2\pi}\iint_{\mathbb{R}^2}e^{-(x^2+y^2)/2}\,\dx\,\dy                   \\
    \end{align*}
    Use polar coordinates to find the final value.
    \begin{align*}
         & =\frac{1}{2\pi}\int_{0}^{\infty}\int_{0}^{2\pi}e^{-r^2/2}r\ \dd\theta\ \dd r \\
         & = 1
    \end{align*}
\end{proof}
{\bf Find the M.G.F of the normal distribution.}


\subsection{Standard Normal Distribution}
\begin{definition}
    A normal distribution with $\mu=0$ and $\sigma=1$ is called {\it standard normal}
    random variable, denoted by $Z$.

    Let X be a normally distributed random variable with mean
    $\mu$ and standard deviation $\sigma$. Then
    \begin{gather}
        Z:=\frac{X-\mu}{\sigma}
    \end{gather}
    has standard normal distribution.
\end{definition}


\subsection{Transformation of Random Variables}
Let $X$ be a continuous random variable with density $f_X$, and $Y=\varphi\circ X$,
where $\varphi:\mathbb{R}\rightarrow\mathbb{R}$ is strictly monotonic and differentiable.
The density for $Y$ is then given by
\begin{gather*}
    f_Y(y)=\begin{cases*}
        f_X(\varphi^{-1}(y))\cdot\left\vert \dfrac{\dd\varphi^{-1}(y)}{\dy}\right\vert & y\in\mathrm{ran }\varphi    \\
        0                                                                              & y\notin\mathrm{ran }\varphi
    \end{cases*}
\end{gather*}

\section{Multivariate Random Variables}
\begin{definition}
    A discrete multivariate random variable is a map that
    \begin{gather*}
        \XX:S\rightarrow\varOmega
    \end{gather*}
\end{definition}

\subsection{The Chebyshev Inequality}
\begin{gather*}
    P[\left\lvert X\right\rvert \ge c]\le\frac{\mathrm{E}[\left\lvert X\right\rvert^k]}{c^k}
\end{gather*}

\subsection{Approximation and Half-Unit Correlation}
\begin{gather*}
    P[X\le y] = \sum_{x=1}^{y}\binom{n}{x}p^x(1-p)^{n-x}\approx\phi\left(\frac{y+1/2-np}{\sqrt{np(1-p)}}\right)
\end{gather*}

% Chapter 9
\section{The Weak Law of Large Numbers}

% Chapter 10
\section{The Hypergeometric Distribution}
Drawing $N$ balls from an urn with $r$ red balls and $N-r$ black balls.
The probability of getting $X$ red balls is given by the hypergeometric distribution.
\begin{definition}
    Let N, n, r $\in\mathbb{N}\backslash\{0\}, r, n\le N$
    A random variable $(X, f_X)$ with
    \begin{gather*}
        X:S\rightarrow\Omega=\{0,\dots, n\}\\
        f_X(x)=\frac{\binom{r}{x}\binom{N-r}{n-x}}{\binom{N}{n}}
    \end{gather*}
    is a hypergeometric distribution with N, n and r.
\end{definition}
\begin{gather*}
    \mathrm{E}[X]=\mathrm{E}[X_1+\cdots+X_n]=\mathrm{E}[X_1]+\cdots+\mathrm{E}[X_n]=n\frac{r}{N}\\
    \mathrm{Var}[X]=\mathrm{Var}[X_1]+\cdots+\mathrm{Var}[X_n]+2\sum_{i<j}^{}\mathrm{Cov}[X_i,X_j]
\end{gather*}
\subsection{Covarience}
\begin{definition}
    \begin{gather*}
        \mathrm{Cov}[X_i,X_j]=\mathrm{E}[X_iX_j]-\mathrm{E}[X_i]\mathrm{E}[X_j]
    \end{gather*}
\end{definition}
We note that $X_iX_j$ is also a Bernoulli variable (with only two outcomes).
Then
\begin{gather*}
    \mathrm{E}[X_iX_j]=p_{ij}:=P[X_i=1\ \mathrm{and}\ X_j=1]
\end{gather*}
Since all Bernoulli trials in the experiment are identical,
\begin{gather*}
    p_{ij} = p_{12}=\frac{r}{N}\cdot\frac{r-1}{N-1}\\
    p_{ii} = p_{11}=\frac{r}{N}.
\end{gather*}
Hence,
\begin{gather*}
    \mathrm{Var}[X_i]=\frac{r}{N}\left(1-\frac{r}{N}\right),\\
    \mathrm{Cov}[X_i,X_j]=-\frac{1}{N}\cdot\frac{r(N-r)}{N(N-1)}.
\end{gather*}

\subsection{Approximating the Hypergeometric Distribution}
An easy calculation gives
\begin{gather*}
    \mathrm{Var}[X]=n\frac{r}{N}\frac{N-r}{N}\frac{N-n}{N-1}.
\end{gather*}
If we let
\begin{gather*}
    p=\frac{r}{N},\quad q=\frac{N-r}{N}
\end{gather*}
the varience given by a binomial distribution is only differed by
\begin{gather*}
    \frac{N-n}{N-1}.
\end{gather*}
The binomial distribution can be used to approximate the hypergeometric distribution
if {\it sampling fraction} $\dfrac{n}{N}$ is less than 0.05.


% Chapter 11
\section{Transformation of Random Variables}
\begin{theorem}
    Let $(\XX,f_{\XX})$ be a continuous multivariate random variable.
    Let $\varphi:\mathbb{R}^n\rightarrow\mathbb{R}^n$
    be a differentiable, bijective map with inverse $\varphi^{-1}$.
    Then $\YY=\varphi\circ\XX$ is a continuous multivariate random variable with
    density
    \begin{gather*}
        f_\YY(y)=f_\XX\circ\varphi^{-1}(y)\cdot\left\lvert \det D\varphi^{-1}(y)\right\rvert     \end{gather*}
\end{theorem}
\begin{lemma}
    Let $((X,Y),f_{XY})$ be a continuous bivariate random variable.
    Let $U=X/Y$, then $f_U$ is given by
    \begin{gather*}
        f_U(u)=\int_{-\infty}^{\infty}f_{XY}(uv,v)\cdot \left\lvert v\right\rvert \ \dd v.
    \end{gather*}
\end{lemma}

\subsection{The Chi Random Variable}

\begin{definition}
    If each $Z_i$ follows a standard normal distribution, the random variable
    \begin{gather*}
        \chi_n:=\sqrt{\sum_{i=1}^{n}Z_i^2}
    \end{gather*}
    follows a chi distribution with $n$ degrees of freedom.
\end{definition}


\begin{lemma}
    Let $\chi^2_{\gamma_1},\dots,\chi^2_{\gamma_n}$ be chi-square distributed independent
    random variables, then
    \begin{gather*}
        \chi^2_\alpha:=\sum_{k=1}^{n}\chi^2_{\gamma_k}
    \end{gather*}
    is a chi-squared random variable with $\alpha=\sum_{k=1}^{n}\gamma_k$
    degrees of freedom.
\end{lemma}

\subsection{The Chi-Squared Distribution}

% Chapter 12
\section{Reliability}
\subsection{The Weibull Density}
\begin{definition}
    A harzard function, with reliability function and
    \begin{gather*}
        \varrho(t)=\alpha\beta t^{\beta-1}\\
        R(t)=e^{-\alpha t^\beta}\\
        f(t)=\varrho(t)R(t)=\alpha\beta t^{\beta-1}e^{-\alpha t^\beta}
    \end{gather*}
\end{definition}


\newpage

\part{Introduction to Stastics}
\section{Samples and Data}
\subsection{Random Sample}
\begin{definition}
    A random sample of size $n$
\end{definition}
The size of sample should not be over 5\%, or additional attention is needed.


\subsection{Percentiles and Quartiles}
The median (second quartile) is a measure of location of the data.
The difference between the third and first quartile is called the
interquartile range,
$$\mathrm{IQR} = q_3 - q_1$$
and is a measure of dispersion of the data.

\subsection{Histogram}
Traditionally, the number of bins $k$ in a histogram, by Sturges should be
\[k=\lceil \log_2n\rceil + 1 \]
However, there are flaws in the

\subsection{The Freedman-Diaconis Rule}
When drawing histograms with fixed bin width, to reduce
\[\delta ^ 2 (h)=\mathrm{E}\left[\int_I \left\lvert H(x)-f_X(x)\right\rvert^2\,\dx \right] \]
so that the error is minimized.
The width $h$, in a simple way of calculation, is given by
\[h=\frac{2\cdot\mathrm{IQR}}{\sqrt[3]{n}}.\]

\subsection{Boxplots}
% p308

\section{Parameter Estimation}
\begin{definition}
    Let $\theta$ be an estimtor, $\hat{\theta}$ be the measured value of the estimator.
    The difference
    \begin{gather*}
        \theta-\mathrm{E}[\hat{\theta}]
    \end{gather*}
    is the bias of the estimtor.

    The mean square error of $\hat{\theta}$ is defined as
    \begin{align*}
        \mathrm{MSE}[\hat{\theta}]: & =\mathrm{E}[(\hat{\theta}-\theta)^2] \\
                                    & =\mathrm{Var}[\hat{\theta}]+(bias)^2
    \end{align*}
\end{definition}
\begin{theorem}
    Let $\Bar{X}$ be the sample mean of a random sample of size n
    from a distribution with mean $\mu$ and variance $\sigma^2$.
    \begin{gather*}
        \mathrm{Var}[\Bar{X}]=\mathrm{E}[(\bar{X}-\mu)^2]=\frac{\sigma^2}{n}
    \end{gather*}
\end{theorem}
\begin{definition}
    The standard deviation of $\Bar{X}$ is given by
    \begin{gather*}
        \sqrt{\mathrm{Var}[\Bar{X}]}=\frac{\sigma}{\sqrt{n}}
    \end{gather*}
    and is called {\bf the standard error of the mean}.
\end{definition}

\subsection{The Method of Moments}
\begin{theorem}
    For any integer $k\ge1$,
    \begin{gather*}
        \widehat{\mathrm{E}[X^k]}=\frac{1}{n}\sum_{i=1}^{n}X_i^k
    \end{gather*}
    is an unbiased estimator for the $k$th moment of $X$.
\end{theorem}

\subsection{Sample Variance}
\begin{definition}
    The sample varience is defined as
    \begin{gather*}
        S^2:=\frac{1}{n-1}\sum_{k=1}^{n}\left(X_k-\bar{X}\right)^2.
    \end{gather*}
    It is also an unbiased estimator.
\end{definition}

\subsection{Method of Maximum Likelihood}
An approach to find estimators.
\begin{definition}
    \begin{gather*}
        L(\theta)=\prod_{i=1}^n f_{X_\theta}(x_i)
    \end{gather*}
    If $X_\theta$ is discrete, $L(\theta)$ is the probability of $(X_1,\dots,X_n)=(x_1,\dots,x_n)$.\\
    If $X_\theta$ is continuous, it is the probability density.\\

    \noindent Then we maximize $L(\theta)$ and let the maximum location to be $\hat{\theta}$.
\end{definition}


% chapter 15

\section{Interval Estimation}
\begin{definition}
    Let $0\le\alpha\le1$, the $100(1-\alpha)\%$ confidence interval for
    parameter $\theta$ is $[L_1, L_2]$ so that
    \begin{gather*}
        P[L_1\le\theta\le L_2]=1-\alpha
    \end{gather*}
\end{definition}

\subsection{Interval Estimation for the Mean (Variance Known)}
We would like to find a number $L$ so that
\begin{gather*}
    P[\bar{X}-L\le\mu\le\bar{X}+L]=1-\alpha
\end{gather*}
The point $z_{\alpha/2}$ is defined by
\begin{gather*}
    \alpha/2=P[Z\ge z_{\alpha/2}]=\frac{1}{\sqrt{2\pi}}\int_{z_{\alpha/2}}^{\infty}
    e^{-x^2/2}\,\dx
\end{gather*}

\subsection{The Helmert Transformation}

\subsection{The $T$-Distribution}
\begin{definition}
    \begin{gather}
        T_\gamma = \frac{Z}{\sqrt{\chi^2_\gamma/\gamma}}
    \end{gather}
\end{definition}


% --------- End of Midterm --------- 

% chapter 16

\section{The Fisher Test}

% chapter 17
\section{Neyman-Pearson Decision Theory}

% chapter 18
\section{Null Hypothesis Significance Testing}
% chapter 19
\section{Single Sample Tests for the Mean and Variance}
% chapter 20
\section{Non-Parametric Single Sample Tests for the Median}
% chapter 21
\section{Inferences on Proportions}
% chapter 22
\section{Comparison of Two Variances}
% chapter 23
\section{Comparison of Two Means}
% chapter 24
\section{Non-Parametric Comparisons; Paired Tests and Correlation}
% chapter 25
\section{Categorical Data}


\newpage
\part{Introduction to Linear Regression}
% chapter 26
\section{Simple Linear Regression I: Basic Model and Inferences}
% chapter 27
\section{Simple Linear Regression II: Predictions and Model Analysis}
% chapter 28
\section{Multiple Linear Regression I: Basic Model}
% chapter 29
\section{Multiple Linear Regression II: Inferences on the Model}
% chapter 30
\section{Multiple Linear Regression III: Finding the Right Model}
% chapter 31
\section{ANOVA I: Basic Model}
% chapter 32
\section{ANOVA II: Homoscedasticity and Post-Tests}


\end{document}
